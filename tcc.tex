\documentclass[
	% -- opções da classe memoir --
	12pt,				% tamanho da fonte
	openright,			% capítulos começam em pág ímpar (insere página vazia caso preciso)
	oneside,			% para impressão em verso e anverso. Oposto a oneside
	a4paper,			% tamanho do papel. 
	% -- opções da classe abntex2 --
	%chapter=TITLE,		% títulos de capítulos convertidos em letras maiúsculas
	%section=TITLE,		% títulos de seções convertidos em letras maiúsculas
	%subsection=TITLE,	% títulos de subseções convertidos em letras maiúsculas
	%subsubsection=TITLE,% títulos de subsubseções convertidos em letras maiúsculas
	% -- opções do pacote babel --
	english,			% idioma adicional para hifenização
	brazil				% o último idioma é o principal do documento
	]{abntex2}

% ----
% Pacotes básicos 
% ----
\usepackage{unitins}
\usepackage{lmodern}			% Usa a fonte Latin Modern			
\usepackage[T1]{fontenc}		% Selecao de codigos de fonte.
\usepackage[utf8]{inputenc}		% Codificacao do documento (conversão automática dos acentos)
\usepackage{lastpage}			% Usado pela Ficha catalográfica
\usepackage{indentfirst}		% Indenta o primeiro parágrafo de cada seção.
\usepackage{color}				% Controle das cores

\usepackage{graphicx}			% Inclusão de gráficos
\usepackage{subfig}
\usepackage{microtype} 			% para melhorias de justificação
\usepackage{soul}
\usepackage{amssymb}
\usepackage{amsmath}
\usepackage{spreadtab}
\usepackage{multirow}
\usepackage{amsthm}
\usepackage{url}

%Adicionado para criar Matrizes, cases e layouts no latex
\usepackage{array}
\usepackage{mathdots}

\usepackage{float}

\usepackage[portuguese, ruled, linesnumbered]{algorithm2e}


%\usepackage{algorithmic}

% Utilizado para customizar as fontes das equações 
\usepackage{fixmath}
% \usepackage{verbatim}
% ---	
% ---
% Pacotes adicionais, usados apenas no âmbito do Modelo Canônico do abnteX2
% ---
		% para geração de dummy text
% ---
% ---
% Pacotes de citações
% ---

%\usepackage[brazilian,hyperpageref]{backref}	 % Paginas com as citações na bibl
\usepackage[alf,abnt-etal-list=2,abnt-repeated-author-omit=yes]{abntex2cite}	% Citações padrão ABNT

\usepackage{multirow}
\usepackage[table,xcdraw]{xcolor}
\usepackage{colortbl}
\definecolor{lightgray}{gray}{0.9}
\graphicspath{{imagens/}}

%INCLUDE PDF 

\usepackage{pdfpages}

% LONG TABLE - TABELA LONGA 

%\usepackage[margin=1in]{geometry}
\usepackage{longtable,tabularx,ltxtable}

\usepackage{threeparttablex} % for "ThreePartTable" environment
%\usepackage{pgfplotstable}
\usepackage{filecontents}
\usepackage{makecell}
\usepackage{booktabs}
\usepackage{float}

\usepackage[brazil]{babel} % for European Portuguese use portuguese
\usepackage[fixlanguage]{babelbib}
\selectbiblanguage{brazil}
\citeoption{abnt-etal-cite=2}


\usepackage[running]{lineno}


% ---
% Informações de dados para CAPA e FOLHA DE ROSTO
% ---
\titulo{Blockchain como Ferramenta de Transparência e Eficiência em Processos de Licitação Pública}
\autor{Erick Azevedo Sousa}
\local{Palmas - TO}
\data{2023}
\orientador{Prof. Me. José Itamar Mendes de Souza Júnior}
\instituicao{%
Universidade Estadual do Tocantins - UNITINS
  \par
  Curso de Sistemas de Informação
  }
\tipotrabalho{Trabalho de Conclusão de Curso (Graduação)}

% O preambulo deve conter o tipo do trabalho, o objetivo, 
% o nome da instituição e a área de concentração 
\preambulo{Projeto apresentado ao Curso de Sistemas de Informação da Fundação Universidade do Tocantins - UNITINS como parte dos requisitos para a obtenção do grau de Bacharel em Sistemas de Informação, sob a orientação do professor Me. José Itamar Mendes de Souza Júnior.}

% ---
% Configurações de aparência do PDF final

% alterando o aspecto da cor azul
\definecolor{blue}{RGB}{41,5,195}

% informações do PDF
\makeatletter
\hypersetup{
     	%pagebackref=true,
		pdftitle={\@title}, 
		pdfauthor={\@author},
    	pdfsubject={\imprimirpreambulo},
		%pdfcreator={LaTeX with abnTeX2},
		pdfkeywords={Métodos mono e multiobjetivos, Técnicas de otimização, Branch and Bound}, 
		colorlinks=true,       		% false: boxed links; true: colored links
    	linkcolor=blue,          	% color of internal links
    	citecolor=blue,        		% color of links to bibliography
    	filecolor=magenta,      		% color of file links
		urlcolor=blue,
		bookmarksdepth=4
}
\makeatother
% --- 

% --- 
% Espaçamentos entre linhas e parágrafos 
% --- 

% O tamanho do parágrafo é dado por:
\setlength{\parindent}{1.3cm}

% Controle do espaçamento entre um parágrafo e outro:
\setlength{\parskip}{0.2cm}  % tente também \onelineskip

% ---
% compila o indice
% ---
\makeindex
% ---

% ----
% Início do documento
% ----
\begin{document}

% Retira espaço extra obsoleto entre as frases.
\frenchspacing

% ----------------------------------------------------------
% ELEMENTOS PRÉ-TEXTUAIS
% ----------------------------------------------------------
% \pretextual
% ----------------------------------------------------------

% ----------------------------------------------------------
% ELEMENTOS PRÉ-TEXTUAIS
% ----------------------------------------------------------
% \pretextual

% ---
% Capa
% ---
\imprimircapa
% ---

% ---
% Folha de rosto
% (o * indica que haverá a ficha bibliográfica)
% ---
\imprimirfolhaderosto
% ---

% ---
% Inserir folha de aprovação
% ---

% Isto é um exemplo de Folha de aprovação, elemento obrigatório da NBR
% 14724/2011 (seção 4.2.1.3). Você pode utilizar este modelo até a aprovação
% do trabalho. Após isso, substitua todo o conteúdo deste arquivo por uma
% imagem da página assinada pela banca com o comando abaixo:
%
% \includepdf{folhadeaprovacao_final.pdf}
% \includepdf[pages={1}]{anexos/Folhadeaprovacao.pdf}
%
\begin{comment}
\begin{folhadeaprovacao}

  	\begin{center}
  		\includegraphics[width=1\textwidth]{imagens/unitins.png}
  		\ABNTEXchapterfont\Large   CURSO DE SISTEMAS DE INFORMA{\c{C}}{\~{A}}O
  		
  		\par
  		\vspace*{.5cm}     
  		{\ABNTEXchapterfont\bfseries\large \expandafter\MakeUppercase{\imprimirtitulo}  \vspace*{1cm}    }
  		\par
  		{\large \expandafter\MakeUppercase{\imprimirautor}}
  		%\vspace*{\fill}
  		\par
  		\vspace*{.5cm}     
  		\hspace{.45\textwidth}
  		\begin{minipage}{.5\textwidth}
  			\small\imprimirpreambulo
  			
  		\end{minipage}%
  	%	\vspace*{\fill}
  	\end{center}
  
   \assinatura{\textbf{\imprimirorientador} \\ Orientador} 
   \assinatura{\textbf{___} \\ Examinador}
   \assinatura{\textbf{___} \\ Examinador}
   %\assinatura{\textbf{___} \\ Convidado 3}
   %\assinatura{\textbf{___} \\ Convidado 4}
      
   \begin{center}
    \vspace*{1cm}
    {\large\imprimirlocal}
    \par
    \vspace*{0.1cm}
    {\large\imprimirdata}
    \vspace*{0.5cm}
  \end{center}
  
\end{folhadeaprovacao}
\end{comment}
% ---

%
\input{Documentos/capitulos/pre-textuais/ficha_catalografica}
\input{Documentos/capitulos/pre-textuais/folha_aprovacao}

% \includepdf[pages={1}]{anexos/AVALIACAO.pdf}
% \includepdf[pages={1}]{anexos/FICHA_CATOLOGRAFICA.pdf}

% ---
% Dedicatória
% ---
\begin{dedicatoria}
   \vspace*{\fill}
   \centering
   \noindent
   \textit{
   a
   } \vspace*{\fill}
\end{dedicatoria}
% ---

% ---
% Agradecimentos
% --- Es
\begin{agradecimentos}

    Gostaria de expressar minha profunda gratidão e reconhecimento à minha família,
que sempre esteve presente e me apoiou incondicionalmente em minha jornada acadêmica
e pessoal. Durante este caminho, enfrentei muitos desafios, incluindo a pandemia global
que afetou significativamente a vida de todos, bem a mudança de cidade, o que tornou tudo ainda mais difícil. Sem o apoio amoroso e
encorajador de minha família, certamente teria sido muito mais difícil chegar até aqui.

    Não posso deixar de expressar minha gratidão ao meu orientador José Itamar Mendes de Souza Júnior, que dedicou inúmeras horas de seu tempo e energia para me guiar neste processo de pesquisa. Sua vasta experiência e conhecimento foram fundamentais para o sucesso deste trabalho, e seu feedback crítico e construtivo me ajudou a aprimorar minhas habilidades e a desenvolver uma dissertação sólida e coerente
	


\end{agradecimentos}
% ---

% ---
% Epígrafe
% ---
\begin{epigrafe}
    \vspace*{\fill}
	\begin{flushright}
		\textit{"Disciplina e paciência são dois fatores fundamentais ao investir"}\\Luiz Barsi
	\end{flushright}
\end{epigrafe}
% ---

% ---
% RESUMOS
% ---

% resumo em português
\setlength{\absparsep}{18pt} 
\begin{resumo}

Português

\textbf{Palavras-chaves}: .
\end{resumo}

% resumo em inglês
\begin{resumo}[Abstract]
 \begin{otherlanguage*}{english}


Ingles
 
	\vspace{\onelineskip}
	\noindent 
	\textbf{Key-words}: .
 \end{otherlanguage*}
\end{resumo}


% ---
% inserir lista de ilustrações
% ---
\include{Documentos/capitulos/pre-textuais/ilustracoes}
% ---

% ---
% inserir lista de tabelas
% ---
\newpage
\pdfbookmark[0]{\listtablename}{lot}
\newpage
\listoftables*
\cleardoublepage
% ---

% ---
% inserir lista de abreviaturas e siglas
% ---
\begin{siglas}
  \item \textbf{A} - \emph{B}.
\end{siglas}
% ---


% ---
% inserir o sumario
% ---
\include{Documentos/capitulos/pre-textuais/sumario}
% ---



% ELEMENTOS TEXTUAIS
% ---------------------------------[!htb]-------------------------
%\textual

%Capitulos
% ----------------------------------------------------------
% Introdução (exemplo de capítulo sem numeração, mas presente no Sumário)
% ----------------------------------------------------------
\chapter{Introdução}\label{cap:intro}
Introdução

\pagebreak

\section{Motivação}\label{session:motivacao}

Motivação

\section{Objetivos}\label{session:objetivos}


\subsection{Objetivo geral }\label{session:objetivo_geral}

Aplicar a tecnologia blockchain de forma eficaz no contexto das licitações públicas, visando aprimorar a transparência, a segurança e a eficiência desses processos, contribuindo para uma gestão mais confiável e eficaz dos recursos públicos.


\subsection{Objetivos específicos}

\begin{itemize}
\item Descrever os conceitos básicos da tecnologia blockchain.
\item Identificar e documentar os pelo menos 5 problemas enfrentados nos processos de licitação.
\item Identificar pelo menos 5 propriedades da blockchain relevantes para a transparência e segurança em licitações.
\item Identificar pelo menos 3 marcos regulatórios  que podem impactar a adoção de blockchain em processos governamentais.
\item Propor um plano estruturado para a integração da tecnologia blockchain em pelo menos 4 fases chave  do processo licitatório.
\item Discutir as barreiras à adoção em larga escala de blockchain em licitações públicas.
\item  Identificar pelo menos 4  possíveis  obstáculos tecnológicos, financeiros e culturais que podem dificultar a implementação.
\end{itemize}

% ---
% Capitulo de revisão de literatura
% ---

\chapter{Referencial Teórico}\label{cap:referencial_teorico}

% Aqui colocar os conceitos, deixar as ferramentas como o framework para a metodologia.


% trabalhos relacionados
\input{Documentos/capitulos/referencial/sections/trabalhos_relacionados}

% materias relacionadas
\input{Documentos/capitulos/referencial/sections/materias_relacionadas}
% ---
% Capitulo de METODOLOGIA
% ---

\chapter{Metodologia}\label{cap:metodologia}

%Colocar somente aqui a parte da crp tecnologia

A pesquisa é do tipo aplicada, de abordagem quantitativa, com objetivo exploratório e de método dedutivo. Os termos são explicados abaixo:

% Avaliacao de desempenho

% ---
% Capitulo de METODOLOGIA
% ---

\chapter{Metodologia}\label{cap:metodologia}

%Colocar somente aqui a parte da crp tecnologia

A pesquisa é do tipo aplicada, de abordagem quantitativa, com objetivo exploratório e de método dedutivo. Os termos são explicados abaixo:

% Avaliacao de desempenho

% ---
% Capitulo de METODOLOGIA
% ---

\chapter{Metodologia}\label{cap:metodologia}

%Colocar somente aqui a parte da crp tecnologia

A pesquisa é do tipo aplicada, de abordagem quantitativa, com objetivo exploratório e de método dedutivo. Os termos são explicados abaixo:

% Avaliacao de desempenho

\input{Documentos/capitulos/metodologia/avaliacoes_de_desempenho/index}

%% Ferramentas utilizadas
\input{Documentos/capitulos/metodologia/ferramentas}

% Definicao de tecnologias

\input{Documentos/capitulos/metodologia/definicao_tecnologias}

%% Ferramentas utilizadas
\section{Ferramentas utilizadas}

\noindent Nessa seção será apresentado as ferramentas utilizadas nessa pesquisa:
\begin{itemize}
\item Visual Studio Code para o desenvolvimento do projeto.
\end{itemize}

% Definicao de tecnologias

\section{Definição das tecnologias}

\subsection{A}


%% Ferramentas utilizadas
\section{Ferramentas utilizadas}

\noindent Nessa seção será apresentado as ferramentas utilizadas nessa pesquisa:
\begin{itemize}
\item Visual Studio Code para o desenvolvimento do projeto.
\end{itemize}

% Definicao de tecnologias

\section{Definição das tecnologias}

\subsection{A}

\chapter{Resultados}\label{cap:resultados}

%Introducao
\chapter{Introdução}\label{cap:intro}
Introdução

%Apresentação dos resultados
\input{Documentos/capitulos/resultados/sections/apresentacao_resultados}
%\subsection{Validação junto com a CRP}
%relatorio do QA
\chapter{Conclusão}\label{cap:conclusao}



% Trabalhos futuros

\section{Trabalhos futuros}

Como trabalhos futuros pretende-se:

\begin{itemize}
    \item a
\end{itemize}


% ----------------------------------------------------------
% ELEMENTOS PÓS-TEXTUAIS
% ----------------------------------------------------------
\postextual
% ----------------------------------------------------------

% ----------------------------------------------------------
% Referências bibliográficas
% ----------------------------------------------------------
%\bibliography{bibliografia}
\bibliography{DOUGLAS_BIBLIOGRAFIA,OUTROS_ARTIGOS,TCC_ALUNOS_DOUGLAS}
% ----------------------------------------------------------
% Glossário
% ----------------------------------------------------------
%
% Consulte o manual da classe abntex2 para orientações sobre o glossário.
%
%\glossary
\include{c_apendice}
%---------------------------------------------------------------------
% INDICE REMISSIVO
%---------------------------------------------------------------------
\phantompart
\printindex
%---------------------------------------------------------------------

\end{document}
\grid
